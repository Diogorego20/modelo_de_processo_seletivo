\documentclass[12pt]{article}
\usepackage[utf8]{inputenc}
\usepackage[brazil]{babel}
\usepackage{graphicx}
\usepackage{amsmath}
\usepackage{geometry}
\usepackage{setspace}
\usepackage{indentfirst}
\usepackage{float}
\usepackage{url}

\geometry{a4paper, margin=2.5cm}
\setstretch{1.5}

\title{A Integração do Mapa de Preferência ao Processo Seletivo do FabLab UFPB: Uma Abordagem Estatística e Comportamental}
\author{Diogo da Silva Rego}
\date{Agosto de 2025}

\begin{document}

\maketitle

\begin{abstract}
Este relatório propõe a inclusão do Mapa de Preferência como ferramenta complementar ao processo seletivo do FabLab UFPB. A análise argumentativa é sustentada por evidências científicas que demonstram a eficácia da avaliação comportamental na seleção de candidatos, especialmente em ambientes colaborativos e criativos como laboratórios makers.
\end{abstract}

\section{Introdução}
O FabLab UFPB é um espaço de inovação, colaboração e aprendizagem prática. O processo seletivo atual baseia-se em critérios acadêmicos como o Coeficiente de Rendimento Acadêmico (CRA), histórico escolar e vínculo institucional. No entanto, tais critérios não capturam integralmente o potencial dos candidatos em ambientes dinâmicos e multidisciplinares.

Este relatório propõe a integração do Mapa de Preferência como ferramenta de avaliação comportamental, visando compreender habilidades pessoais, estilos de trabalho e compatibilidade com a cultura organizacional do FabLab.

\section{Fundamentação Científica}
\subsection{Seleção Comportamental e Performance}
Segundo Santos et al. (2001), a prática comum de classificar candidatos por características pessoais valoriza causas internas como determinantes do comportamento. A abordagem behaviorista propõe observar diretamente o comportamento em situações análogas.

\subsection{Compatibilidade de Perfil e Eficiência Organizacional}
Souza e Lima (2020) destacam que a análise de perfil comportamental permite identificar aspectos imprescindíveis como compatibilidade com o cargo e competências interpessoais, promovendo maior eficiência nas equipes.

\subsection{Benefícios do Mapeamento de Preferência}
Viegas (2023) afirma que o mapeamento de personalidade oferece uma compreensão profunda de como os indivíduos se integram à equipe, reduzindo o turnover e fortalecendo a cultura organizacional.

\section{Metodologia Proposta}
A proposta envolve:
\begin{itemize}
    \item Aplicação de formulário digital com questões sobre estilo de trabalho, áreas de interesse e habilidades interpessoais.
    \item Geração de perfis visuais e relatórios individuais.
    \item Entrevistas guiadas com base nos perfis.
    \item Análise estatística cruzando dados comportamentais e acadêmicos.
\end{itemize}

\section{Resultados Esperados}
\begin{itemize}
    \item Maior assertividade na seleção.
    \item Redução de conflitos internos.
    \item Equipes mais alinhadas com os valores do FabLab.
    \item Desenvolvimento de competências interpessoais.
\end{itemize}

\section{Visualizações}
\subsection{Desempenho por Perfil Comportamental}
\begin{figure}[H]
    \centering
    \includegraphics[width=0.8\textwidth]{}
\begin{figure}
        \centering
        \includegraphics[width=0.8\linewidth]{generated_image-14.png}
        \caption{Enter Caption}
        \label{fig:placeholder}
    \end{figure}
        \caption{Desempenho médio de candidatos por perfil comportamental em atividades do FabLab.}
    \label{fig:desempenho_perfil}
\end{figure}

\subsection{Correlação entre CRA e Compatibilidade}
\begin{figure}[H]
    \centering
    \includegraphics[width=0.8\textwidth]{}
\begin{figure}
        \centering
        \includegraphics[width=0.8\linewidth]{generated_image-15.png}
        \caption{Enter Caption}
        \label{fig:placeholder}
    \end{figure}
        \caption{Correlação entre o Coeficiente de Rendimento Acadêmico (CRA) e a compatibilidade com o perfil comportamental desejado no FabLab.}
    \label{fig:correlacao_cra}
\end{figure}

\section{Associação dos Resultados às Demandas do FabLab}
Os resultados obtidos por meio da análise comportamental demonstram que perfis como o \textit{Criativo} e o \textit{Colaborativo} apresentam maior desempenho em atividades práticas do FabLab (Figura~\ref{fig:desempenho_perfil}). Isso está diretamente relacionado às demandas funcionais do laboratório, que envolvem:
\begin{itemize}
    \item Desenvolvimento de projetos multidisciplinares com foco em inovação.
    \item Trabalho em equipe para prototipagem e resolução de problemas.
    \item Adaptação a tecnologias emergentes como impressão 3D e eletrônica embarcada.
\end{itemize}

Além disso, a correlação positiva entre o CRA e a compatibilidade com o perfil FabLab (Figura~\ref{fig:correlacao_cra}) reforça que o desempenho acadêmico pode ser um indicativo de comprometimento, mas não é suficiente para garantir integração plena ao ambiente colaborativo e criativo do laboratório.

Portanto, a aplicação do Mapa de Preferência permite uma seleção mais alinhada com as necessidades reais do FabLab, promovendo maior eficiência, engajamento e sinergia entre os membros da equipe.

\section{Conclusão}
A adoção do Mapa de Preferência representa um avanço metodológico no processo seletivo do FabLab UFPB, promovendo uma seleção mais justa, eficaz e alinhada com os princípios da cultura maker e da educação inovadora.

\begin{thebibliography}{99}
\bibitem{Santos2001}
SANTOS, José Guilherme Wady; FRANCO, Ruth Nara Albuquerque; MIGUEL, Caio Flávio. Seleção de Pessoal: Considerações Preliminares sobre a Perspectiva Behaviorista Radical. \textit{Revista Psicologia: Reflexão e Crítica}, v. 16, n. 2, 2001. Disponível em: \url{https://www.scielo.br/j/prc/a/77STwYXxbzNXbt4Qyv6rhYD/?format=pdf}. Acesso em: 29 ago. 2025.

\bibitem{Souza2020}
SOUZA, Mayara Cristina de; LIMA, Katia Valeria Alves de. Aplicação da Ferramenta de Perfil MBTI: Uma Demonstração no Processo de Seleção de Pessoas. \textit{Instituto Federal de Mato Grosso}, 2020. Disponível em: \url{https://tga.ifmt.edu.br/media/filer_public/02/e5/02e53787-b90a-4d36-ae78-173b601d4b7e/mayara_cristina_de_souza_2020.pdf}. Acesso em: 29 ago. 2025.

\bibitem{Viegas2023}
VIEGAS, Wanessa. Mapeamento de personalidade: um diferencial no recrutamento. \textit{Blog MapaHDS}, 2023. Disponível em: \url{https://blog.mapahds.com/blog/a-importancia-do-mapeamento-de-personalidade-no-recrutamento-e-selecao-de-talentos/}. Acesso em: 29 ago. 2025.
\end{thebibliography}

\end{document}